\chapter{Experimental Results} \label{chap:meth}

\newcolumntype{?}{!{\vrule width 1.5pt}}


\section{Datasets}\label{datasets}


\subsection{MNIST+P Dataset}\label{data-mnistp}
The MNIST database of handwritten digits (10 classes) has a training set of 60,000 examples, and a test set of 10,000 examples, where wach image is of dimensions 28x28. 


It is a commonly used benchmark for supervised learning algorithm performance. In the MNIST+P dataset created for our experimentation, we consider that an image of MNIST dataset represents the magnitude of a complex number. The magnitude in each image is scaled such that the black part of the image obtains the value of 50, and the white part is 200. The phase for each image class is picked from uniform distribution whose range is determined class-wise: The range for class 1 representing the digit 0 is [0,$\pi$/10], that for class 2 representing the digit 1 is [$\pi$/10,2$\pi$/10],...., that for class 10 representing digit 9 is [9$\pi$/10,$\pi$]. 

FIGURE:

\subsubsection{Preprocessing of MNIST+P Dataset}
Now that we know the magnitude and phase information for each image, we convert them to the cartesian form having a real channel and an imaginary channel and set the shape of each image becomes 32x32x2.





\subsection{Radar datasets}\label{data-radar}
   In our case, a triangular chirp is being produced at frequency of 150 GHz, with a bandwidth of 6 GHz
\subsection{Radar-150 datasets}
\subsection{Radar-300 datasets}
The magnitude and phase maps are first reshaped to a same size: 32x32. 


\subsubsection{Preprocessing of Radar Dataset}
We convert the Polar form information (magnitude and phase) to Cartesian form having a real channel and an imaginary channel. Now, the shape of each image becomes 32x32x2.


How to write about polar and cartesian form?


\subsection{Data Analysis}


\section{Results}

%\begin{table}
\begin{center}
	\captionof{table}{Test accuracy (\%) of activation functions in $\mathbb{C}$-CNNs (z$ReLU(z)$, $\mathbb{C}$ReLU(z)), $tanh(z)$), $\mathbb{R}$-CNNs (ReLU) on MNIST+P dataset}
	\begin{tabular}{ c|c|c|c?c } 
		%\hline
		- &$z$ReLU(z) & $\mathbb{C}$ReLU(z) & tanh($z$) & ReLU\\
		\hline Test Accuracy (\%) & 98.31 & 99.07 & 88.42&\textbf{99.59}\\
		%\hline
	\end{tabular}

\end{center}
%\end{table}

Peculiar problem with class with digit 1

\begin{center}
		\captionof{table}{Test accuracy (\%) of activation functions in $\mathbb{C}$-CNNs (z$ReLU(z)$, $\mathbb{C}$ReLU(z)), $tanh(z)$) and $\mathbb{R}$-CNNs (ReLU) on Cartesian and Polar representation of Radar-150 and Radar-300 datasets}
	\begin{tabular}{ c|c|c|c?c } 
		%\hline
		- &$z$ReLU(z) & $\mathbb{C}$ReLU(z) & tanh($z$) & ReLU\\
		\hline Radar-150-Cart (\%) & - & - & - & -\\
		\hline Radar-150-Cart (\%) & - & - & - & -\\
		\hline Radar-300-Polar (\%) & - & - & - & -\\
		\hline Radar-300-Polar (\%) & - & - & - & -\\
		
	\end{tabular}
\end{center}

\subsection{Training acc curves}
\subsection{Test loss cruves}
\section{Discussion}
We didn't experience convergence difficulties.
Generalization what?
Perform at par


