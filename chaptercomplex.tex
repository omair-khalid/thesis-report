\chapter{Background}\label{chap:comp}
\section{Radar Technology?}
\section{Complex Domain Theory}
\subsection{Complex Numbers}
representation
hello
\subsection{Complex Functions}




Representation types
DO I DESCRIBE THEIR IMPLEMENTATION HERE?

\subsection{Holomorphism \& Complex Differentiability}

Holomorphism guarantees that a complex-valued function is complex differentiable in the neighborhood of every point in its domain \cite{trabelsi2018deep}.
A complex function $f(z): \mathbb{C} \mapsto \mathbb{C}$, $f(z)= u(z) + iv(z)$ ($u(z)$ and $v(z)$ are real-valued functions) is complex differentiable if it satisfies the following two conditions:
\begin{enumerate}

	\item It satisfies the Cauchy-Riemann (CR) equations, given by:
\begin{equation}\label{eq:compdiff}
\frac{\delta u}{\delta x} = \frac{\delta v}{\delta y} ~~,~~ \frac{\delta u}{\delta y} = - \frac{\delta v}{\delta x} 
\end{equation}
	\item $u(z)$ and $v(z)$ are individually differentiable at $z$ as real functions	
\end{enumerate} 

While CR equations are a necessary condition for holomorphism, the presence of the second condition makes CR equations become a sufficient condition for holomorphism.  