\chapter{Methodology} \label{chap:methodology}
 
 WHY BRO WHY DID YOU CHOOSE AVERAGE POOLING?
 AHAAA!  We don't wanna have to deal with a activation function whose values are only positive!
 
 breifly mention the challenges of building a complex valued neural networks. 
 
 \section{Representation of complex numbers}
 Consider a complex number $z=a+ib$ with the real component $a$ and the imaginary component $b$. In $\mathbb{R}$-CNNs, the filter bank or the feature maps of a layer exist in 3 dimensions, where two of them denote the size of the filter or feature map in 2D, and the third dimension indicates how many there are. If we have $N$ feature maps (where $N$ is divisible by 2), the first $N/2$ feature maps would be dedicated to the real components ($a$) and the last $N/2$ feature maps would be dedicated to the imaginary components ($b$). In this manner, the imaginary feature map corresponding to the real feature map on index one of the 3rd dimension would like on the $\frac{N}{2} +1$ position.
 
 IMAGE REQUIRED
 
 \section{Experimental setups}
 Explain the architecture, experimental details, the datasets used. Datasets: Toy+ results.Tables and shit.
 
 \subsection{MNIST+P Dataset}
 
 \subsection{Radar Dataset}
 
 \subsection{Archtitecture - $\mathbb{R}$-CNNs \& $\mathbb{C}$-CNNs}
 
 
 