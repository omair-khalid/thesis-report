\chapter{Introduction} \label{chap:intro}

\section{Motivation} \label{sect:thefirst}
Motivation in LeCUN

Why CVNN?

Page 38 akira 2012
 In CVNNs, the flexibility in learning and self-organization
is restricted rather than that in double-dimensional real-valued neural net-
works. As we discussed in Section 3.2.1, the restriction is brought by the four
fundamental rules of arithmetic in complex numbers (especially multiplication as it entails rotation and scaling, not just scaling as in real valued case). Such fundamental rules
in processing often work well in solving real world problems. This feature is
one of the most useful advantages in CVNNs. wave-related phenomena
such as sonic wave, lightwave, and electromagnetic wave.4

pg 19
In short, in CVNNs, we can reduce ineffective degree of freedom in
learning or self-organization to achieve better generalization characteristics.

 if we know a priori
that the objective quantities include “phase” and/or “amplitude,” we can re-
duce possibly harmful portion of the freedom by employing a complex-valued
neural network, resulting in a more meaningful generalization characteristics\\

physical meaning of radar magnitude and phase? discernibility lies in them?

the mannerism of how the learning unfolds is restricted


biological motivation
signal processing
associative memory
data patchhhhh
historical uses
success with PolSAR

MAKE SECTIONS
 



\section{Problem definition} \label{sect:thefirst}
complex data with complex weights?
leveraging the promise of complex valued networks

\section{Document structure} \label{sect:thefirst}
% % You are strongly encouraged to use the Latex templates provided.

% \subsection{Paper}
% The manuscript should be in A4 size, and the printed paper should
% be of at least 70 gsm.

% \subsection{Font and margins}
% Thesis should be printed on both sides of the paper. Use no less
% than 1.5 spacing, with quotations and notes single-spaced.
% Regarding \textbf{Character size}, not less than 2.0mm for
% capitals and 1.5mm for x-height (the height of a lower-case x). Us
% a serif font (i.e. Times) between 10 and 12 points. Use consistent
% and clear fonts through all the document.

% The text layout should be approximately as follows:

% \begin{itemize}
%     \item $4cm$ binding margin
%     \item $2cm$ head margin (top of page)
%     \item $2.5cm$ fore-edge margin
%     \item $4cm$ tail margin (bottom of page)
% \end{itemize}

% \section{Title Page}
% The title page should contain the title of thesis, authors name,
% and at the foot of the page: the name of degree,  Your University,
% and the year of presentation. Something like this:

% \vspace*{1cm}
% \begin{center}
% {\Large\bf MSc. Thesis example VIBOT\\} \vspace{2cm} {\large
% Robert Mart\'i\\
% \vspace{1cm}
% Department of Computer Architecture and Technology \\
% University of Girona}

% \end{center}

% \vspace{2cm}
% \begin{center}
% {\large A Thesis Submitted for the Degree of MSc Erasmus Mundus in
% Vision and Robotics (VIBOT)\\ \vspace{0.3cm} $\cdot$ 2008 $\cdot$}
% \end{center}