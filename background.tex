\chapter{Background}\label{chap:comp}
\section{Radar Technology?}


\section{Complex Domain Theory}

\subsection{Complex Numbers and its representaitons}
\subsubsection{Polar form}
\subsubsection{Cartesian form}
\subsubsection{Exponential form}
\subsection{Complex functions}
map c to c
map c to r



Representation types
DO I DESCRIBE THEIR IMPLEMENTATION HERE?

\subsection{Holomorphism \& Complex Differentiability}

Holomorphism guarantees that a complex-valued function is complex differentiable in the neighborhood of every point in its domain \cite{trabelsi2018deep}.
A complex function $f(z): \mathbb{C} \mapsto \mathbb{C}$, $f(z)= u(z) + iv(z)$ ($u(z)$ and $v(z)$ are real-valued functions) is complex differentiable if it satisfies the following two conditions:
\begin{enumerate}

	\item It satisfies the Cauchy-Riemann (CR) equations, given by:
\begin{equation}\label{eq:compdiff}
\frac{\delta u}{\delta x} = \frac{\delta v}{\delta y} ~~,~~ \frac{\delta u}{\delta y} = - \frac{\delta v}{\delta x} 
\end{equation}
	\item $u(z)$ and $v(z)$ are individually differentiable at $z$ as real functions	
\end{enumerate} 

\subsection{Derivative of real-valued complex functions}

 
\subsection{Generalized chain rule for real-valued complex loss  functions}\label{cchainrule}
Consider that we have a real-valued loss function $L(z): \mathbb{C}\rightarrow\mathbb{R}$, where $z$ is a complex variable $z=x+iy$ with $x,y \in \mathbb{R}$. The gradient of real-valued complex function is defined as follows:
 
\begin{equation}\label{cvgrad}
\nabla_{L}(z) = \frac{\partial L}{\partial z} = \frac{\partial L}{\partial x} + i\frac{\partial L}{\partial y} = \Re(\nabla_{L}(z))+i\Im(\nabla_{L}(z))
\end{equation}

If we have another complex variable $t=r+is$ where z could be expressed in terms of $t$, where $r,s \in \mathbb{R}$, the gradient with respect to $t$ would be:

\begin{equation}\label{chainchainchain}
\begin{align*}
\nabla_{L}(t) &= \frac{\partial L}{\partial t} = \frac{\partial L}{\partial r} + i\frac{\partial L}{\partial s}\\
&= \frac{\partial L}{\partial x}\frac{\partial x}{\partial r} + \frac{\partial L}{\partial y}\frac{\partial y}{\partial r}+i\bigg(\frac{\partial L}{\partial x}\frac{\partial x}{\partial s}+\frac{\partial L}{\partial y}\frac{\partial y}{\partial s}\bigg)\\
&= \frac{\partial L}{\partial x}\bigg(\frac{\partial x}{\partial r}+i\frac{\partial x}{\partial s}\bigg)+\frac{\partial L}{\partial y}\bigg(\frac{\partial y}{\partial r}+i\frac{\partial y}{\partial s}\bigg)\\
&= \Re(\nabla_{L}(z))\bigg(\frac{\partial x}{\partial r}+i\frac{\partial x}{\partial s}\bigg)+\Im(\nabla_{L}(z))\bigg(\frac{\partial y}{\partial r}+i\frac{\partial y}{\partial s}\bigg)
\end{align*}
\end{equation}






