\chapter{Background}\label{chap:comp}
\section{Radar technology?}


\section{Complex numbers}

\subsection{Complex numbers and their representations}
Complex numbers

The field of complex numbers includes the field of real numbers as a subfield.

Complex numbers are useful abstract quantities that can be used in calculations and result in physically meaningful solutions.
quantity i




\subsubsection{Cartesian form}
In the Cartesian form, a complex number is represented by $z=x+iy$, where $x$ is the imaginary part and $y$ is the real part. They can be visualized in a rectangular plane, called the Complex Plane, where the x-axis represents the real part and the y-axis represents the imaginary part (demonstrated in Figure BLABLA). 

\subsubsection{Polar form}
The Polar form of a complex number can be denoted by two quantities: 
the distance to a point (x,y) from the origin on the complex plane, $|z|$, and the phase of the complex number, $\theta$, measured positive counter-clockwise from the x-axis. The conversion formulae to and from Cartesian form and Polar form are given by:

\begin{equation}
x = |z|cos\theta  ~~,~~ y = |z|sin\theta 
\end{equation}

\begin{equation}
|z| = \sqrt{x_{2}+y_{2}}  ~~,~~ \theta = \arctan(y/x) 
\end{equation}

The magnitude or distance quantity $|z|$ is unique for each $(x,y)$ in the complex domain, but the $\theta$ can have the following values: $\theta$, $\theta \pm \mathrm{2}\pi$,...,, $\theta \pm \mathrm{2}n\pi$ where $n$ = 1,2,3,... However, we often choose a valued between $0$ and 2$\pi$ and call it the \textit{Principal Argument}.

\begin{figure}[htb]
	\centering
	\epsfxsize=5cm
	{\epsfbox{crep2}}
	\caption{Illustration of complex number in Cartesian form and Polar form on a Complex Plane}
	\label{fig:complexrep}
\end{figure}

\subsubsection{Complex numbers operations}
\paragraph{Conjugate}
The complex conjugate of a complex number $z=x+iy$ is given by $x-iy$, denoted by $z^{*}$ or $\bar{z}$. Conjugate of a complex number can be used to extract the real and imaginary parts of a complex number, as follows:

\begin{equation}
\begin{aligned}
\Re(z)= \frac{z+\bar{z}}{2}\\
\Im(z)= \frac{z-\bar{z}}{2}
\end{aligned}
\end{equation}

\paragraph{Multiplication}
Let $z = a+ib$ and $t=c+id$ be two complex numbers. The product of $z$ and $t$ would be as follows:
\begin{equation}
\begin{aligned}
(a+ib)(c+id)  &= ac+iad+ibc-bd\\
&= (ac-bd)+i(ad+bc)
\end{aligned}
\end{equation}

\paragraph{Addition and Substraction}
Let $z = a+ib$ and $t=c+id$ be two complex numbers. The addition and substraction of $z$ and $t$ would be as follows:

\begin{equation}
\begin{aligned}
(a+ib) + (c+id)  = (a+c)+i(b+d)\\
(a+ib) - (c+id)  = (a-c)+i(b-d)
\end{aligned}
\end{equation}

As can be seen, the addition and substraction operations are performed 
component-wise.

\paragraph{Division}
Algebraic closed field

$\theta$





\subsubsection{Exponential form}
\subsection{Complex functions}


\subsection{Properties of complex number field}
map c to c
map c to r



\subsection{Holomorphism \& Complex Differentiability}

Holomorphism guarantees that a complex-valued function is complex differentiable in the neighborhood of every point in its domain \cite{trabelsi2018deep}.
A complex function $f(z): \mathbb{C} \mapsto \mathbb{C}$, $f(z)= u(z) + iv(z)$ ($u(z)$ and $v(z)$ are real-valued functions) is complex differentiable if it satisfies the following two conditions:
\begin{enumerate}

	\item It satisfies the Cauchy-Riemann (CR) equations, given by:
\begin{equation}\label{eq:compdiff}
\frac{\delta u}{\delta x} = \frac{\delta v}{\delta y} ~~,~~ \frac{\delta u}{\delta y} = - \frac{\delta v}{\delta x} 
\end{equation}
	\item $u(z)$ and $v(z)$ are individually differentiable at $z$ as real functions	
\end{enumerate} 

\subsection{Generalized chain rule for real-valued complex loss  functions}\label{cchainrule}
Consider that we have a real-valued loss function $L(z): \mathbb{C}\rightarrow\mathbb{R}$, where $z$ is a complex variable $z=x+iy$ with $x,y \in \mathbb{R}$. The gradient of real-valued complex function is defined as follows:
 
\begin{equation}\label{cvgrad}
\nabla_{L}(z) = \frac{\partial L}{\partial z} = \frac{\partial L}{\partial x} + i\frac{\partial L}{\partial y} = \Re(\nabla_{L}(z))+i\Im(\nabla_{L}(z))
\end{equation}

If we have another complex variable $t=r+is$ where z could be expressed in terms of $t$, where $r,s \in \mathbb{R}$, the gradient with respect to $t$ would be:

\begin{equation}\label{chainchainchain}
\begin{aligned}
\nabla_{L}(t) &= \frac{\partial L}{\partial t} = \frac{\partial L}{\partial r} + i\frac{\partial L}{\partial s}\\
&= \frac{\partial L}{\partial x}\frac{\partial x}{\partial r} + \frac{\partial L}{\partial y}\frac{\partial y}{\partial r}+i\bigg(\frac{\partial L}{\partial x}\frac{\partial x}{\partial s}+\frac{\partial L}{\partial y}\frac{\partial y}{\partial s}\bigg)\\
&= \frac{\partial L}{\partial x}\bigg(\frac{\partial x}{\partial r}+i\frac{\partial x}{\partial s}\bigg)+\frac{\partial L}{\partial y}\bigg(\frac{\partial y}{\partial r}+i\frac{\partial y}{\partial s}\bigg)\\
&= \Re(\nabla_{L}(z))\bigg(\frac{\partial x}{\partial r}+i\frac{\partial x}{\partial s}\bigg)+\Im(\nabla_{L}(z))\bigg(\frac{\partial y}{\partial r}+i\frac{\partial y}{\partial s}\bigg)
\end{aligned}
\end{equation}






