\chapter{State of the art}\label{chap:comp}

history

rnn

Guberman

Synthetic Aperture Radars is an imaging technology

 
hansch 2010 hansch 2010 zhang wide sar atr

The incorporation of complex-valued blocks in the classification of Synthetic Aperture Radar (SAR) data has been very promising. 

H\"{a}nsch and Hellwich (2009) extended the use of s




The same authors



Zhang $et \ al.$ (2017) \cite{polsarzhang2017complex} leverage magnitude as well as the phase of the PolSAR data to classify different terrains on the Flevoland (3 classes) and Oberpfaffenhofen datasets (15 classes). Compared to $\mathbb{R}$-CNN, $\mathbb{C}$-CNN performs better on both the datasets while having approximately same number of parameters.  



Chiheb $et \ al.$ (2018) \cite{trabelsi2018deep} compare the performance of different architectures of the $\mathbb{C}$-CNNs and $\mathbb{R}$-CNNs on the tasks of image recognition, music transcription, and speech spectrum prediction. $\mathbb{C}$-CNNs were reported to perform comparably to $\mathbb{R}$-CNNs for the first task, and achieve state-of-the-art performance on the two tasks while beating $\mathbb{R}$-CNNs. The authors also contribute the extension of Batch Normalization and Weight Initialization (BN) to complex domain. 